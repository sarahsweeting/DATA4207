% Options for packages loaded elsewhere
\PassOptionsToPackage{unicode}{hyperref}
\PassOptionsToPackage{hyphens}{url}
\PassOptionsToPackage{dvipsnames,svgnames,x11names}{xcolor}
%
\documentclass[
]{article}
\usepackage{amsmath,amssymb}
\usepackage{iftex}
\ifPDFTeX
  \usepackage[T1]{fontenc}
  \usepackage[utf8]{inputenc}
  \usepackage{textcomp} % provide euro and other symbols
\else % if luatex or xetex
  \usepackage{unicode-math} % this also loads fontspec
  \defaultfontfeatures{Scale=MatchLowercase}
  \defaultfontfeatures[\rmfamily]{Ligatures=TeX,Scale=1}
\fi
\usepackage{lmodern}
\ifPDFTeX\else
  % xetex/luatex font selection
  \setmainfont[]{Avenir Book}
\fi
% Use upquote if available, for straight quotes in verbatim environments
\IfFileExists{upquote.sty}{\usepackage{upquote}}{}
\IfFileExists{microtype.sty}{% use microtype if available
  \usepackage[]{microtype}
  \UseMicrotypeSet[protrusion]{basicmath} % disable protrusion for tt fonts
}{}
\makeatletter
\@ifundefined{KOMAClassName}{% if non-KOMA class
  \IfFileExists{parskip.sty}{%
    \usepackage{parskip}
  }{% else
    \setlength{\parindent}{0pt}
    \setlength{\parskip}{6pt plus 2pt minus 1pt}}
}{% if KOMA class
  \KOMAoptions{parskip=half}}
\makeatother
\usepackage{xcolor}
\usepackage[margin=1in]{geometry}
\usepackage{color}
\usepackage{fancyvrb}
\newcommand{\VerbBar}{|}
\newcommand{\VERB}{\Verb[commandchars=\\\{\}]}
\DefineVerbatimEnvironment{Highlighting}{Verbatim}{commandchars=\\\{\}}
% Add ',fontsize=\small' for more characters per line
\usepackage{framed}
\definecolor{shadecolor}{RGB}{248,248,248}
\newenvironment{Shaded}{\begin{snugshade}}{\end{snugshade}}
\newcommand{\AlertTok}[1]{\textcolor[rgb]{0.94,0.16,0.16}{#1}}
\newcommand{\AnnotationTok}[1]{\textcolor[rgb]{0.56,0.35,0.01}{\textbf{\textit{#1}}}}
\newcommand{\AttributeTok}[1]{\textcolor[rgb]{0.13,0.29,0.53}{#1}}
\newcommand{\BaseNTok}[1]{\textcolor[rgb]{0.00,0.00,0.81}{#1}}
\newcommand{\BuiltInTok}[1]{#1}
\newcommand{\CharTok}[1]{\textcolor[rgb]{0.31,0.60,0.02}{#1}}
\newcommand{\CommentTok}[1]{\textcolor[rgb]{0.56,0.35,0.01}{\textit{#1}}}
\newcommand{\CommentVarTok}[1]{\textcolor[rgb]{0.56,0.35,0.01}{\textbf{\textit{#1}}}}
\newcommand{\ConstantTok}[1]{\textcolor[rgb]{0.56,0.35,0.01}{#1}}
\newcommand{\ControlFlowTok}[1]{\textcolor[rgb]{0.13,0.29,0.53}{\textbf{#1}}}
\newcommand{\DataTypeTok}[1]{\textcolor[rgb]{0.13,0.29,0.53}{#1}}
\newcommand{\DecValTok}[1]{\textcolor[rgb]{0.00,0.00,0.81}{#1}}
\newcommand{\DocumentationTok}[1]{\textcolor[rgb]{0.56,0.35,0.01}{\textbf{\textit{#1}}}}
\newcommand{\ErrorTok}[1]{\textcolor[rgb]{0.64,0.00,0.00}{\textbf{#1}}}
\newcommand{\ExtensionTok}[1]{#1}
\newcommand{\FloatTok}[1]{\textcolor[rgb]{0.00,0.00,0.81}{#1}}
\newcommand{\FunctionTok}[1]{\textcolor[rgb]{0.13,0.29,0.53}{\textbf{#1}}}
\newcommand{\ImportTok}[1]{#1}
\newcommand{\InformationTok}[1]{\textcolor[rgb]{0.56,0.35,0.01}{\textbf{\textit{#1}}}}
\newcommand{\KeywordTok}[1]{\textcolor[rgb]{0.13,0.29,0.53}{\textbf{#1}}}
\newcommand{\NormalTok}[1]{#1}
\newcommand{\OperatorTok}[1]{\textcolor[rgb]{0.81,0.36,0.00}{\textbf{#1}}}
\newcommand{\OtherTok}[1]{\textcolor[rgb]{0.56,0.35,0.01}{#1}}
\newcommand{\PreprocessorTok}[1]{\textcolor[rgb]{0.56,0.35,0.01}{\textit{#1}}}
\newcommand{\RegionMarkerTok}[1]{#1}
\newcommand{\SpecialCharTok}[1]{\textcolor[rgb]{0.81,0.36,0.00}{\textbf{#1}}}
\newcommand{\SpecialStringTok}[1]{\textcolor[rgb]{0.31,0.60,0.02}{#1}}
\newcommand{\StringTok}[1]{\textcolor[rgb]{0.31,0.60,0.02}{#1}}
\newcommand{\VariableTok}[1]{\textcolor[rgb]{0.00,0.00,0.00}{#1}}
\newcommand{\VerbatimStringTok}[1]{\textcolor[rgb]{0.31,0.60,0.02}{#1}}
\newcommand{\WarningTok}[1]{\textcolor[rgb]{0.56,0.35,0.01}{\textbf{\textit{#1}}}}
\usepackage{graphicx}
\makeatletter
\def\maxwidth{\ifdim\Gin@nat@width>\linewidth\linewidth\else\Gin@nat@width\fi}
\def\maxheight{\ifdim\Gin@nat@height>\textheight\textheight\else\Gin@nat@height\fi}
\makeatother
% Scale images if necessary, so that they will not overflow the page
% margins by default, and it is still possible to overwrite the defaults
% using explicit options in \includegraphics[width, height, ...]{}
\setkeys{Gin}{width=\maxwidth,height=\maxheight,keepaspectratio}
% Set default figure placement to htbp
\makeatletter
\def\fps@figure{htbp}
\makeatother
\setlength{\emergencystretch}{3em} % prevent overfull lines
\providecommand{\tightlist}{%
  \setlength{\itemsep}{0pt}\setlength{\parskip}{0pt}}
\setcounter{secnumdepth}{-\maxdimen} % remove section numbering
\usepackage{caption}
\captionsetup[figure]{font=scriptsize}
\ifLuaTeX
  \usepackage{selnolig}  % disable illegal ligatures
\fi
\IfFileExists{bookmark.sty}{\usepackage{bookmark}}{\usepackage{hyperref}}
\IfFileExists{xurl.sty}{\usepackage{xurl}}{} % add URL line breaks if available
\urlstyle{same}
\hypersetup{
  pdftitle={Group exercise 2: Analysing spatial data},
  pdfauthor={Emily and Sarah},
  colorlinks=true,
  linkcolor={Maroon},
  filecolor={Maroon},
  citecolor={Blue},
  urlcolor={blue},
  pdfcreator={LaTeX via pandoc}}

\title{Group exercise 2: Analysing spatial data}
\usepackage{etoolbox}
\makeatletter
\providecommand{\subtitle}[1]{% add subtitle to \maketitle
  \apptocmd{\@title}{\par {\large #1 \par}}{}{}
}
\makeatother
\subtitle{DATA5207: Data Analysis in the Social Sciences}
\author{Emily and Sarah}
\date{}

\begin{document}
\maketitle

\hypertarget{introduction}{%
\section{Introduction}\label{introduction}}

We will be examining the relationship between opiate drug prescriptions
and poverty (your independent variables), and mortality caused by drugs
and alcohol using the methods covered in this lab, and previously in
this unit. The spatial aspect of this is that we are going to use
county-level data sourced from the *US Census Bureau* and the *Centers
for Disease Control and Prevention (CDC)*.

We have included two additional variables from county.data:

\begin{itemize}
\item
  State of each county: Because different states in the US have
  different laws which may affect access to drugs and alcohol
\item
  Education - Percent of adults with a high school diploma only 2014-18:
  We believe this is a confounding factor of poverty and mortality from
  drugs since education will affect the jobs you are able to obtain
  (leading to poverty if unobtainable) and educated drug usage (which we
  are assuming gets taught in high school in the US). If they haven't
  been to high school, there is a more likely abuse of drugs since they
  are uneducated on the effects on their drug usage. Similarly, with
  poverty, if they are uneducated, they are less likely to have the
  skills to get a job and be able to obtain an income and then are more
  likely to enter poverty.
\end{itemize}

\hypertarget{data-cleaning}{%
\subsection{Data cleaning}\label{data-cleaning}}

Firstly, we shall load all the data files.

\begin{Shaded}
\begin{Highlighting}[]
\FunctionTok{load}\NormalTok{(}
  \StringTok{"US data for group task/Other census data/county.data.RData"}\NormalTok{)}
\end{Highlighting}
\end{Shaded}

\begin{Shaded}
\begin{Highlighting}[]
\NormalTok{poverty.data }\OtherTok{\textless{}{-}} \FunctionTok{read.csv}\NormalTok{(}
  \StringTok{"US data for group task/poverty/poverty data {-} ACS\_16\_5YR\_S1702\_with\_ann.csv"}\NormalTok{, }
  \AttributeTok{skip =} \DecValTok{1}\NormalTok{, }\AttributeTok{header =}\NormalTok{ T)}
\end{Highlighting}
\end{Shaded}

\begin{Shaded}
\begin{Highlighting}[]
\NormalTok{drugs.mort.data }\OtherTok{\textless{}{-}} \FunctionTok{read.table}\NormalTok{(}
  \StringTok{"US data for group task/mortality/Drugs and alcohol Cause of Death, 1999{-}2016.txt"}\NormalTok{, }
    \AttributeTok{sep =} \StringTok{\textquotesingle{}}\SpecialCharTok{\textbackslash{}t}\StringTok{\textquotesingle{}}\NormalTok{, }\AttributeTok{header =}\NormalTok{ T)}
\end{Highlighting}
\end{Shaded}

\begin{Shaded}
\begin{Highlighting}[]
\NormalTok{opioids.data }\OtherTok{\textless{}{-}} \FunctionTok{read.csv}\NormalTok{(}\StringTok{"US data for group task/opioids/opioids prescribing rates.csv"}\NormalTok{)}
\end{Highlighting}
\end{Shaded}

\begin{Shaded}
\begin{Highlighting}[]
\NormalTok{us.shape}\FloatTok{.1} \OtherTok{\textless{}{-}} 
  \FunctionTok{st\_read}\NormalTok{(}\StringTok{"US data for group task/US shapefiles/cb\_2016\_us\_county\_20m.shp"}\NormalTok{)}
\end{Highlighting}
\end{Shaded}

\begin{verbatim}
## Reading layer `cb_2016_us_county_20m' from data source 
##   `C:\Users\emily\Documents\GitHub\DATA4207-GroupProject\group-project-2\US data for group task\US shapefiles\cb_2016_us_county_20m.shp' 
##   using driver `ESRI Shapefile'
## Simple feature collection with 3220 features and 9 fields
## Geometry type: MULTIPOLYGON
## Dimension:     XY
## Bounding box:  xmin: -179.1743 ymin: 17.91377 xmax: 179.7739 ymax: 71.35256
## Geodetic CRS:  NAD83
\end{verbatim}

\begin{Shaded}
\begin{Highlighting}[]
\NormalTok{fip.concordance }\OtherTok{\textless{}{-}} \FunctionTok{read.csv}\NormalTok{(}\StringTok{"US data for group task/fips concordance.csv"}\NormalTok{)}
\end{Highlighting}
\end{Shaded}

\vspace{6mm}

We will join datasets on their similar column - the FIPs code for the
county:

\begin{Shaded}
\begin{Highlighting}[]
\FunctionTok{head}\NormalTok{(poverty.data}\SpecialCharTok{$}\NormalTok{Id2)}
\end{Highlighting}
\end{Shaded}

\begin{verbatim}
## [1] 1001 1003 1005 1007 1009 1011
\end{verbatim}

\begin{Shaded}
\begin{Highlighting}[]
\FunctionTok{head}\NormalTok{(drugs.mort.data}\SpecialCharTok{$}\NormalTok{County.Code)}
\end{Highlighting}
\end{Shaded}

\begin{verbatim}
## [1] 1001 1003 1005 1007 1009 1011
\end{verbatim}

\begin{Shaded}
\begin{Highlighting}[]
\FunctionTok{head}\NormalTok{(county.data}\SpecialCharTok{$}\NormalTok{FIPS)}
\end{Highlighting}
\end{Shaded}

\begin{verbatim}
## [1]    0 1000 1001 1003 1005 1007
\end{verbatim}

\begin{Shaded}
\begin{Highlighting}[]
\FunctionTok{head}\NormalTok{(opioids.data}\SpecialCharTok{$}\NormalTok{FIPS.County.Code)}
\end{Highlighting}
\end{Shaded}

\begin{verbatim}
## [1] 2013 2016 2020 2050 2060 2068
\end{verbatim}

\vspace{6mm}

\begin{Shaded}
\begin{Highlighting}[]
\NormalTok{patterns }\OtherTok{\textless{}{-}} \FunctionTok{c}\NormalTok{(}\StringTok{\textquotesingle{} County| Parish| city| Borough| Census Area\textquotesingle{}}\NormalTok{)}

\NormalTok{main.data }\OtherTok{\textless{}{-}} \FunctionTok{merge}\NormalTok{(}
\NormalTok{  poverty.data }\SpecialCharTok{\%\textgreater{}\%} 
\NormalTok{    dplyr}\SpecialCharTok{::}\FunctionTok{rename}\NormalTok{(}\AttributeTok{below.poverty.rate =} 
\NormalTok{             All.families....Percent.below.poverty.level..Estimate..Families)  }\SpecialCharTok{\%\textgreater{}\%} 
\NormalTok{    dplyr}\SpecialCharTok{::}\FunctionTok{select}\NormalTok{(Id2, below.poverty.rate),}
\NormalTok{  drugs.mort.data }\SpecialCharTok{\%\textgreater{}\%} 
\NormalTok{    dplyr}\SpecialCharTok{::}\FunctionTok{rename}\NormalTok{(}\AttributeTok{Id2 =}\NormalTok{ County.Code,}
                \AttributeTok{mortality.rate =}\NormalTok{ Crude.Rate) }\SpecialCharTok{\%\textgreater{}\%} 
    \FunctionTok{mutate}\NormalTok{(}\AttributeTok{county =} \FunctionTok{gsub}\NormalTok{(patterns, }\StringTok{""}\NormalTok{, County))  }\SpecialCharTok{\%\textgreater{}\%}
\NormalTok{    dplyr}\SpecialCharTok{::}\FunctionTok{select}\NormalTok{(Id2, county, mortality.rate))}
\end{Highlighting}
\end{Shaded}

\begin{Shaded}
\begin{Highlighting}[]
\NormalTok{main.data }\OtherTok{\textless{}{-}} \FunctionTok{merge}\NormalTok{(}
\NormalTok{  main.data }\SpecialCharTok{\%\textgreater{}\%}
\NormalTok{    dplyr}\SpecialCharTok{::}\FunctionTok{select}\NormalTok{(Id2, county, mortality.rate, below.poverty.rate),}
\NormalTok{  county.data }\SpecialCharTok{\%\textgreater{}\%}
\NormalTok{    dplyr}\SpecialCharTok{::}\FunctionTok{rename}\NormalTok{(}\AttributeTok{Id2 =}\NormalTok{ FIPS,}
                 \AttributeTok{county.state =}\NormalTok{ State,}
                 \AttributeTok{education.highschool =}\NormalTok{ Percent.of.adults.with.less.than.a.high.school.diploma.}\FloatTok{2014.18}\NormalTok{) }\SpecialCharTok{\%\textgreater{}\%}
    \FunctionTok{mutate}\NormalTok{(}\AttributeTok{z.highschool =}\NormalTok{ (education.highschool }\SpecialCharTok{{-}} \FunctionTok{mean}\NormalTok{(education.highschool, }\AttributeTok{na.rm=}\NormalTok{T)) }\SpecialCharTok{/}
           \FunctionTok{sd}\NormalTok{(education.highschool, }\AttributeTok{na.rm=}\NormalTok{T)) }\SpecialCharTok{\%\textgreater{}\%}
\NormalTok{    dplyr}\SpecialCharTok{::}\FunctionTok{select}\NormalTok{(Id2, county.state, education.highschool, z.highschool)}
\NormalTok{)}
\end{Highlighting}
\end{Shaded}

\begin{Shaded}
\begin{Highlighting}[]
\NormalTok{main.data }\OtherTok{\textless{}{-}} \FunctionTok{merge}\NormalTok{(}
\NormalTok{  main.data  }\SpecialCharTok{\%\textgreater{}\%}
  \FunctionTok{mutate}\NormalTok{(}\AttributeTok{z.poverty =}\NormalTok{ (below.poverty.rate }\SpecialCharTok{{-}} \FunctionTok{mean}\NormalTok{(below.poverty.rate, }\AttributeTok{na.rm=}\NormalTok{T)) }\SpecialCharTok{/} 
           \FunctionTok{sd}\NormalTok{(below.poverty.rate, }\AttributeTok{na.rm=}\NormalTok{T)), }
\NormalTok{  opioids.data }\SpecialCharTok{\%\textgreater{}\%}
\NormalTok{  dplyr}\SpecialCharTok{::}\FunctionTok{rename}\NormalTok{(}\AttributeTok{Id2 =}\NormalTok{ FIPS.County.Code) }\SpecialCharTok{\%\textgreater{}\%} 
  \FunctionTok{mutate}\NormalTok{(}\AttributeTok{z.opiates =}\NormalTok{ (X2016.Prescribing.Rate }\SpecialCharTok{{-}} \FunctionTok{mean}\NormalTok{(X2016.Prescribing.Rate, }\AttributeTok{na.rm=}\NormalTok{T)) }\SpecialCharTok{/}
           \FunctionTok{sd}\NormalTok{(X2016.Prescribing.Rate, }\AttributeTok{na.rm=}\NormalTok{T)) }\SpecialCharTok{\%\textgreater{}\%}
\NormalTok{    dplyr}\SpecialCharTok{::}\FunctionTok{select}\NormalTok{(Id2, z.opiates, X2016.Prescribing.Rate))}
\end{Highlighting}
\end{Shaded}

Viewing the first six rows of our new data frame, we can see the
structure of these data: \vspace{6mm}

\begin{verbatim}
##    Id2      county mortality.rate below.poverty.rate county.state
## 1 1001 Autauga, AL           12.5                9.4           AL
## 2 1003 Baldwin, AL           22.6                9.3           AL
## 3 1005 Barbour, AL            9.0               20.0           AL
## 4 1007    Bibb, AL           14.1               11.7           AL
## 5 1009  Blount, AL           18.1               12.2           AL
## 6 1011 Bullock, AL           11.1               25.3           AL
##   education.highschool z.highschool     z.poverty  z.opiates
## 1                 11.3   -0.3612033 -0.4937185239  1.2662069
## 2                  9.7   -0.6019626 -0.5113760968  1.1283754
## 3                 27.0    2.0012473  1.3779842063  0.3893133
## 4                 16.8    0.4664068 -0.0875943466  0.4962515
## 5                 19.8    0.9178304  0.0006935181 -0.4614399
## 6                 24.8    1.6702032  2.3138355714 -1.3430863
##   X2016.Prescribing.Rate
## 1                  129.6
## 2                  123.8
## 3                   92.7
## 4                   97.2
## 5                   56.9
## 6                   19.8
\end{verbatim}

The first row is the ID variable we used to merge our data frames. The
below.poverty.rate is the percentage of each county's families with
incomes below the poverty rate. county, the name of each country (and
the state within which they sit) while as county.state is the state
code. X2016.Prescribing.Rate is the rate of opiod prescriptions.
education.highschool is the percentage that completed highschool only.
Finally, the mortality rate associated with drug and alcohol use (per
100,000 people).

In this code, we also standardise the education, poverty and
prescription rates in the same line of block of code that we use to
merge the data.

There is one last edit to make to these data before we examine them. If
you run the code

\begin{Shaded}
\begin{Highlighting}[]
\FunctionTok{levels}\NormalTok{(}\FunctionTok{as.factor}\NormalTok{(main.data}\SpecialCharTok{$}\NormalTok{mortality.rate))}
\end{Highlighting}
\end{Shaded}

\begin{verbatim}
##   [1] "10.0"       "10.1"       "10.2"       "10.3"       "10.4"      
##   [6] "10.5"       "10.6"       "10.7"       "10.8"       "10.9"      
##  [11] "11.0"       "11.1"       "11.2"       "11.3"       "11.4"      
##  [16] "11.5"       "11.6"       "11.7"       "11.8"       "11.9"      
##  [21] "113.4"      "12.0"       "12.1"       "12.2"       "12.3"      
##  [26] "12.4"       "12.5"       "12.6"       "12.7"       "12.8"      
##  [31] "12.9"       "13.0"       "13.1"       "13.2"       "13.3"      
##  [36] "13.4"       "13.5"       "13.6"       "13.7"       "13.8"      
##  [41] "13.9"       "14.0"       "14.1"       "14.2"       "14.3"      
##  [46] "14.4"       "14.5"       "14.6"       "14.7"       "14.8"      
##  [51] "14.9"       "15.0"       "15.1"       "15.2"       "15.3"      
##  [56] "15.4"       "15.5"       "15.6"       "15.7"       "15.8"      
##  [61] "15.9"       "16.0"       "16.1"       "16.2"       "16.3"      
##  [66] "16.4"       "16.5"       "16.6"       "16.7"       "16.8"      
##  [71] "16.9"       "17.0"       "17.1"       "17.2"       "17.3"      
##  [76] "17.4"       "17.5"       "17.6"       "17.7"       "17.8"      
##  [81] "17.9"       "18.0"       "18.1"       "18.2"       "18.3"      
##  [86] "18.4"       "18.5"       "18.6"       "18.7"       "18.8"      
##  [91] "18.9"       "19.0"       "19.1"       "19.2"       "19.3"      
##  [96] "19.4"       "19.5"       "19.6"       "19.7"       "19.8"      
## [101] "19.9"       "20.0"       "20.1"       "20.2"       "20.3"      
## [106] "20.4"       "20.5"       "20.6"       "20.7"       "20.8"      
## [111] "20.9"       "21.0"       "21.1"       "21.2"       "21.3"      
## [116] "21.4"       "21.5"       "21.6"       "21.7"       "21.8"      
## [121] "21.9"       "22.0"       "22.1"       "22.2"       "22.3"      
## [126] "22.4"       "22.5"       "22.6"       "22.7"       "22.8"      
## [131] "22.9"       "23.0"       "23.1"       "23.2"       "23.3"      
## [136] "23.4"       "23.5"       "23.6"       "23.7"       "23.8"      
## [141] "23.9"       "24.0"       "24.1"       "24.2"       "24.3"      
## [146] "24.4"       "24.5"       "24.6"       "24.7"       "24.8"      
## [151] "24.9"       "25.0"       "25.1"       "25.2"       "25.3"      
## [156] "25.4"       "25.5"       "25.6"       "25.7"       "25.8"      
## [161] "25.9"       "26.0"       "26.1"       "26.2"       "26.3"      
## [166] "26.4"       "26.5"       "26.6"       "26.7"       "26.8"      
## [171] "26.9"       "27.0"       "27.1"       "27.2"       "27.3"      
## [176] "27.4"       "27.5"       "27.6"       "27.7"       "27.8"      
## [181] "27.9"       "28.0"       "28.1"       "28.2"       "28.3"      
## [186] "28.4"       "28.5"       "28.6"       "28.7"       "28.8"      
## [191] "28.9"       "29.0"       "29.1"       "29.2"       "29.4"      
## [196] "29.5"       "29.6"       "29.7"       "29.8"       "29.9"      
## [201] "30.0"       "30.1"       "30.2"       "30.3"       "30.5"      
## [206] "30.6"       "30.7"       "30.8"       "30.9"       "31.0"      
## [211] "31.1"       "31.2"       "31.3"       "31.4"       "31.5"      
## [216] "31.6"       "31.7"       "31.8"       "31.9"       "32.3"      
## [221] "32.4"       "32.5"       "32.6"       "32.7"       "32.8"      
## [226] "32.9"       "33.0"       "33.1"       "33.2"       "33.3"      
## [231] "33.4"       "33.5"       "33.6"       "33.7"       "33.8"      
## [236] "33.9"       "34.0"       "34.1"       "34.2"       "34.3"      
## [241] "34.4"       "34.5"       "34.6"       "34.8"       "34.9"      
## [246] "35.0"       "35.1"       "35.3"       "35.4"       "35.6"      
## [251] "35.7"       "35.9"       "36.0"       "36.1"       "36.2"      
## [256] "36.3"       "36.4"       "36.5"       "36.7"       "36.8"      
## [261] "36.9"       "37.0"       "37.2"       "37.3"       "37.4"      
## [266] "37.5"       "37.7"       "37.9"       "38.0"       "38.1"      
## [271] "38.2"       "38.3"       "38.4"       "38.5"       "38.6"      
## [276] "38.7"       "38.8"       "38.9"       "39.1"       "39.2"      
## [281] "39.4"       "39.5"       "39.6"       "39.7"       "39.9"      
## [286] "4.2"        "4.6"        "4.9"        "40.0"       "40.4"      
## [291] "40.5"       "40.6"       "40.7"       "40.8"       "41.0"      
## [296] "41.2"       "41.5"       "41.7"       "41.8"       "41.9"      
## [301] "42.1"       "42.2"       "42.3"       "42.4"       "42.5"      
## [306] "42.9"       "43.2"       "43.3"       "43.5"       "43.6"      
## [311] "43.9"       "44.1"       "44.2"       "44.5"       "44.6"      
## [316] "44.7"       "44.9"       "45.0"       "45.1"       "45.4"      
## [321] "45.5"       "45.7"       "45.8"       "45.9"       "46.2"      
## [326] "46.7"       "46.9"       "47.0"       "47.5"       "47.9"      
## [331] "48.2"       "48.5"       "49.0"       "49.1"       "49.7"      
## [336] "49.9"       "5.3"        "5.8"        "5.9"        "50.0"      
## [341] "50.1"       "50.2"       "50.3"       "51.0"       "51.4"      
## [346] "51.5"       "51.8"       "51.9"       "52.2"       "52.7"      
## [351] "52.9"       "53.0"       "53.1"       "53.4"       "53.8"      
## [356] "54.5"       "54.6"       "55.0"       "55.7"       "55.8"      
## [361] "56.0"       "58.4"       "59.2"       "6.2"        "6.3"       
## [366] "6.5"        "6.7"        "6.8"        "60.3"       "61.6"      
## [371] "61.8"       "64.3"       "66.8"       "67.2"       "67.3"      
## [376] "67.4"       "68.9"       "7.1"        "7.2"        "7.3"       
## [381] "7.4"        "7.5"        "7.6"        "7.7"        "7.8"       
## [386] "7.9"        "70.1"       "74.5"       "78.9"       "8.0"       
## [391] "8.1"        "8.2"        "8.3"        "8.4"        "8.5"       
## [396] "8.6"        "8.7"        "8.8"        "8.9"        "80.2"      
## [401] "83.7"       "86.3"       "87.1"       "9.0"        "9.1"       
## [406] "9.2"        "9.3"        "9.4"        "9.5"        "9.6"       
## [411] "9.7"        "9.8"        "9.9"        "90.8"       "97.6"      
## [416] "Unreliable"
\end{verbatim}

you will see that some of the observations for the variable
\texttt{mortality.rate} are coded as `Unreliable'. This creates two
issues for us. Those observations are effectively missing for our
purposes, and this converts the variable to a string when we want it to
be a number.

\vspace{6mm}

\begin{Shaded}
\begin{Highlighting}[]
\NormalTok{main.data }\OtherTok{\textless{}{-}}\NormalTok{ main.data }\SpecialCharTok{\%\textgreater{}\%} 
  \FunctionTok{mutate}\NormalTok{(}\AttributeTok{mortality.rate =} \FunctionTok{ifelse}\NormalTok{(mortality.rate }\SpecialCharTok{==} \StringTok{\textquotesingle{}Unreliable\textquotesingle{}}\NormalTok{, }
                                 \ConstantTok{NA}\NormalTok{, }\FunctionTok{as.numeric}\NormalTok{(}\FunctionTok{as.character}\NormalTok{(mortality.rate))))}
\end{Highlighting}
\end{Shaded}

We then want to link these shapefiles to the county results. This file,
which we were required to modify a little above, can be matched to the
shapefile data using similar syntax to above:

\vspace{6mm}

\begin{Shaded}
\begin{Highlighting}[]
\NormalTok{us.shape}\FloatTok{.2} \OtherTok{\textless{}{-}}\NormalTok{ us.shape}\FloatTok{.1} \SpecialCharTok{\%\textgreater{}\%} 
  \FunctionTok{merge}\NormalTok{(fip.concordance }\SpecialCharTok{\%\textgreater{}\%}
          \FunctionTok{mutate}\NormalTok{(}\AttributeTok{STATEFP =}  \FunctionTok{ifelse}\NormalTok{(Numeric.code }\SpecialCharTok{\textgreater{}=} \DecValTok{0} \SpecialCharTok{\&}\NormalTok{ Numeric.code }\SpecialCharTok{\textless{}=} \DecValTok{9}\NormalTok{, }
                             \FunctionTok{paste0}\NormalTok{(}\DecValTok{0}\NormalTok{ , Numeric.code), Numeric.code)) }\SpecialCharTok{\%\textgreater{}\%}
\NormalTok{          dplyr}\SpecialCharTok{::}\FunctionTok{rename}\NormalTok{(}\AttributeTok{state =}\NormalTok{ Alpha.code))}
\end{Highlighting}
\end{Shaded}

We edit the variable containing county names in shapefile using the
\texttt{paste0()} function, which combines the county name with the
state abbreviation (from the fip concordance file). We then modify
county names with some different spelling in the shapefiles and our
county level data:

\vspace{6mm}

\begin{Shaded}
\begin{Highlighting}[]
\NormalTok{us.shape}\FloatTok{.2}\SpecialCharTok{$}\NormalTok{county }\OtherTok{\textless{}{-}} \FunctionTok{paste0}\NormalTok{(us.shape}\FloatTok{.2}\SpecialCharTok{$}\NormalTok{NAME, }\StringTok{", "}\NormalTok{, us.shape}\FloatTok{.2}\SpecialCharTok{$}\NormalTok{state)}

\NormalTok{us.shape}\FloatTok{.2}\SpecialCharTok{$}\NormalTok{county }\OtherTok{\textless{}{-}} \FunctionTok{gsub}\NormalTok{(}\StringTok{"Doña Ana, NM"}\NormalTok{, }\StringTok{"Dona Ana, NM"}\NormalTok{, us.shape}\FloatTok{.2}\SpecialCharTok{$}\NormalTok{county)}
\NormalTok{us.shape}\FloatTok{.2}\SpecialCharTok{$}\NormalTok{county }\OtherTok{\textless{}{-}} \FunctionTok{gsub}\NormalTok{(}\StringTok{"LaSalle, LA"}\NormalTok{, }\StringTok{"La Salle, LA"}\NormalTok{, us.shape}\FloatTok{.2}\SpecialCharTok{$}\NormalTok{county)}
\NormalTok{us.shape}\FloatTok{.2}\SpecialCharTok{$}\NormalTok{county }\OtherTok{\textless{}{-}} \FunctionTok{gsub}\NormalTok{(}\StringTok{"Oglala Lakota, SD"}\NormalTok{, }\StringTok{"Oglala, SD"}\NormalTok{, us.shape}\FloatTok{.2}\SpecialCharTok{$}\NormalTok{county)}
\end{Highlighting}
\end{Shaded}

\vspace{6mm}

And then we merge the shapefile and the county-level data:

\vspace{6mm}

\begin{Shaded}
\begin{Highlighting}[]
\NormalTok{us.shape}\FloatTok{.2} \OtherTok{\textless{}{-}} \FunctionTok{full\_join}\NormalTok{(us.shape}\FloatTok{.2}\NormalTok{,}
\NormalTok{  main.data }\SpecialCharTok{\%\textgreater{}\%}
\NormalTok{    dplyr}\SpecialCharTok{::}\FunctionTok{select}\NormalTok{(county, below.poverty.rate, mortality.rate, X2016.Prescribing.Rate)) }\SpecialCharTok{\%\textgreater{}\%}
  \FunctionTok{st\_as\_sf}\NormalTok{() }\CommentTok{\# To acknolwedge its a shape file}
\end{Highlighting}
\end{Shaded}

\begin{verbatim}
## Joining with `by = join_by(county)`
\end{verbatim}

\begin{verbatim}
## Warning in sf_column %in% names(g): Detected an unexpected many-to-many relationship between `x` and `y`.
## i Row 1205 of `x` matches multiple rows in `y`.
## i Row 1138 of `y` matches multiple rows in `x`.
## i If a many-to-many relationship is expected, set `relationship =
##   "many-to-many"` to silence this warning.
\end{verbatim}

\begin{Shaded}
\begin{Highlighting}[]
\FunctionTok{shift\_geometry}\NormalTok{(us.shape}\FloatTok{.2}\NormalTok{, }\AttributeTok{position=}\StringTok{"below"}\NormalTok{)}
\end{Highlighting}
\end{Shaded}

\begin{verbatim}
## Simple feature collection with 3237 features and 17 fields (with 5 geometries empty)
## Geometry type: GEOMETRY
## Dimension:     XY
## Bounding box:  xmin: -3112200 ymin: -1697728 xmax: 2258154 ymax: 1558935
## Projected CRS: USA_Contiguous_Albers_Equal_Area_Conic
## First 10 features:
##    STATEFP COUNTYFP COUNTYNS       AFFGEOID GEOID       NAME LSAD      ALAND
## 1       01      017 00161534 0500000US01017 01017   Chambers   06 1545009121
## 2       01      029 00161540 0500000US01029 01029   Cleburne   06 1450663941
## 3       01      043 00161547 0500000US01043 01043    Cullman   06 1903376541
## 4       01      105 00161579 0500000US01105 01105      Perry   06 1863919953
## 5       01      097 00161575 0500000US01097 01097     Mobile   06 3184131335
## 6       01      009 00161530 0500000US01009 01009     Blount   06 1670041875
## 7       01      073 00161562 0500000US01073 01073  Jefferson   06 2878224117
## 8       01      023 00161537 0500000US01023 01023    Choctaw   06 2365955057
## 9       01      033 00161542 0500000US01033 01033    Colbert   06 1534878514
## 10      01      077 00161564 0500000US01077 01077 Lauderdale   06 1729334080
##        AWATER    Name state Numeric.code          Status         county
## 1    17048126 Alabama    AL            1 State; counties   Chambers, AL
## 2     2354898 Alabama    AL            1 State; counties   Cleburne, AL
## 3    52118974 Alabama    AL            1 State; counties    Cullman, AL
## 4    10918482 Alabama    AL            1 State; counties      Perry, AL
## 5  1073840930 Alabama    AL            1 State; counties     Mobile, AL
## 6    15077458 Alabama    AL            1 State; counties     Blount, AL
## 7    32446239 Alabama    AL            1 State; counties  Jefferson, AL
## 8    19059247 Alabama    AL            1 State; counties    Choctaw, AL
## 9    80030048 Alabama    AL            1 State; counties    Colbert, AL
## 10  134420924 Alabama    AL            1 State; counties Lauderdale, AL
##    below.poverty.rate mortality.rate X2016.Prescribing.Rate
## 1                15.3           15.7                  140.6
## 2                12.2           11.4                   44.2
## 3                12.0           16.9                  166.2
## 4                29.8             NA                    3.6
## 5                15.6           19.2                  133.5
## 6                12.2           18.1                   56.9
## 7                14.3           24.0                  116.7
## 8                16.2           10.1                   97.8
## 9                13.1            9.9                  208.1
## 10               13.4           14.8                  127.5
##                          geometry
## 1  MULTIPOLYGON (((963232 -438...
## 2  MULTIPOLYGON (((934720.6 -3...
## 3  MULTIPOLYGON (((813867.6 -3...
## 4  MULTIPOLYGON (((789425.1 -5...
## 5  MULTIPOLYGON (((718804.6 -6...
## 6  MULTIPOLYGON (((828860.7 -3...
## 7  MULTIPOLYGON (((804636.5 -4...
## 8  MULTIPOLYGON (((708460.5 -5...
## 9  MULTIPOLYGON (((714661.3 -2...
## 10 MULTIPOLYGON (((704900.9 -2...
\end{verbatim}

\vspace{6mm}

You are now ready to map these data. Good luck with the rest of the
exercise!

\begin{Shaded}
\begin{Highlighting}[]
\NormalTok{main.data }\OtherTok{=} \FunctionTok{drop\_na}\NormalTok{(main.data, mortality.rate, z.poverty, z.opiates, z.highschool)}

\NormalTok{cannabis\_legality }\OtherTok{\textless{}{-}} \FunctionTok{data.frame}\NormalTok{(}
  \AttributeTok{county.state =} \FunctionTok{c}\NormalTok{(}\StringTok{"AL"}\NormalTok{, }\StringTok{"AK"}\NormalTok{, }\StringTok{"AZ"}\NormalTok{, }\StringTok{"AR"}\NormalTok{, }\StringTok{"CA"}\NormalTok{, }
            \StringTok{"CO"}\NormalTok{, }\StringTok{"CT"}\NormalTok{, }\StringTok{"DE"}\NormalTok{, }\StringTok{"FL"}\NormalTok{, }\StringTok{"GA"}\NormalTok{, }
            \StringTok{"HI"}\NormalTok{, }\StringTok{"ID"}\NormalTok{, }\StringTok{"IL"}\NormalTok{, }\StringTok{"IN"}\NormalTok{, }\StringTok{"IA"}\NormalTok{, }
            \StringTok{"KS"}\NormalTok{, }\StringTok{"KY"}\NormalTok{, }\StringTok{"LA"}\NormalTok{, }\StringTok{"ME"}\NormalTok{, }\StringTok{"MD"}\NormalTok{, }
            \StringTok{"MA"}\NormalTok{, }\StringTok{"MI"}\NormalTok{, }\StringTok{"MN"}\NormalTok{, }\StringTok{"MS"}\NormalTok{, }
            \StringTok{"MO"}\NormalTok{, }\StringTok{"MT"}\NormalTok{, }\StringTok{"NE"}\NormalTok{, }\StringTok{"NV"}\NormalTok{, }\StringTok{"NH"}\NormalTok{, }
            \StringTok{"NJ"}\NormalTok{, }\StringTok{"NM"}\NormalTok{, }\StringTok{"NY"}\NormalTok{, }\StringTok{"NC"}\NormalTok{, }
            \StringTok{"ND"}\NormalTok{, }\StringTok{"OH"}\NormalTok{, }\StringTok{"OK"}\NormalTok{, }\StringTok{"OR"}\NormalTok{, }\StringTok{"PA"}\NormalTok{, }
            \StringTok{"RI"}\NormalTok{, }\StringTok{"SC"}\NormalTok{, }\StringTok{"SD"}\NormalTok{, }\StringTok{"TN"}\NormalTok{, }
            \StringTok{"TX"}\NormalTok{, }\StringTok{"UT"}\NormalTok{, }\StringTok{"VT"}\NormalTok{, }\StringTok{"VA"}\NormalTok{, }\StringTok{"WA"}\NormalTok{, }
            \StringTok{"WV"}\NormalTok{, }\StringTok{"WI"}\NormalTok{, }\StringTok{"WY"}\NormalTok{, }\StringTok{"DC"}\NormalTok{),}
  \AttributeTok{Legality\_Status =} \FunctionTok{c}\NormalTok{(}\StringTok{"Legal for Medical Only"}\NormalTok{, }\StringTok{"Legal for Medical and Recreational"}\NormalTok{, }
                      \StringTok{"Legal for Medical and Recreational"}\NormalTok{, }\StringTok{"Legal for Medical Only"}\NormalTok{, }
                      \StringTok{"Legal for Medical and Recreational"}\NormalTok{, }\StringTok{"Legal for Medical and Recreational"}\NormalTok{, }
                      \StringTok{"Legal for Medical and Recreational"}\NormalTok{, }\StringTok{"Legal for Medical Only"}\NormalTok{, }
                      \StringTok{"Legal for Medical Only"}\NormalTok{, }\StringTok{"Legal for Medical Only"}\NormalTok{, }
                      \StringTok{"Legal for Medical Only"}\NormalTok{, }\StringTok{"Illegal"}\NormalTok{, }\StringTok{"Legal for Medical and Recreational"}\NormalTok{, }
                      \StringTok{"Legal for Medical Only"}\NormalTok{, }\StringTok{"Legal for Medical Only"}\NormalTok{, }
                      \StringTok{"Legal for Medical Only"}\NormalTok{, }\StringTok{"Legal for Medical Only"}\NormalTok{, }
                      \StringTok{"Legal for Medical Only"}\NormalTok{, }\StringTok{"Legal for Medical and Recreational"}\NormalTok{, }
                      \StringTok{"Legal for Medical and Recreational"}\NormalTok{, }\StringTok{"Legal for Medical and Recreational"}\NormalTok{, }
                      \StringTok{"Legal for Medical and Recreational"}\NormalTok{, }\StringTok{"Legal for Medical Only"}\NormalTok{, }
                      \StringTok{"Legal for Medical Only"}\NormalTok{, }\StringTok{"Legal for Medical and Recreational"}\NormalTok{, }
                      \StringTok{"Legal for Medical and Recreational"}\NormalTok{, }\StringTok{"Decriminalized"}\NormalTok{, }
                      \StringTok{"Legal for Medical and Recreational"}\NormalTok{, }\StringTok{"Legal for Medical Only"}\NormalTok{, }
                      \StringTok{"Legal for Medical and Recreational"}\NormalTok{, }\StringTok{"Legal for Medical and Recreational"}\NormalTok{, }
                      \StringTok{"Legal for Medical and Recreational"}\NormalTok{, }\StringTok{"Legal for Medical Only"}\NormalTok{, }
                      \StringTok{"Legal for Medical Only"}\NormalTok{, }\StringTok{"Legal for Medical Only"}\NormalTok{, }
                      \StringTok{"Legal for Medical Only"}\NormalTok{, }\StringTok{"Legal for Medical and Recreational"}\NormalTok{, }
                      \StringTok{"Legal for Medical Only"}\NormalTok{, }\StringTok{"Legal for Medical and Recreational"}\NormalTok{, }
                      \StringTok{"Legal for Medical Only"}\NormalTok{, }\StringTok{"Legal for Medical and Recreational"}\NormalTok{, }
                      \StringTok{"Legal for Medical Only"}\NormalTok{, }\StringTok{"Legal for Medical Only"}\NormalTok{, }
                      \StringTok{"Legal for Medical Only"}\NormalTok{, }\StringTok{"Legal for Medical and Recreational"}\NormalTok{, }
                      \StringTok{"Legal for Medical and Recreational"}\NormalTok{, }\StringTok{"Legal for Medical and Recreational"}\NormalTok{, }
                      \StringTok{"Legal for Medical Only"}\NormalTok{, }\StringTok{"Legal for Medical Only"}\NormalTok{, }\StringTok{"Illegal"}\NormalTok{, }
                      \StringTok{"Legal for Medical and Recreational"}\NormalTok{)}
\NormalTok{)}

\NormalTok{legalplus.main.data }\OtherTok{=} \FunctionTok{merge}\NormalTok{(main.data, cannabis\_legality, }\AttributeTok{by =} \StringTok{"county.state"}\NormalTok{, }\AttributeTok{all =} \ConstantTok{TRUE}\NormalTok{)}
\end{Highlighting}
\end{Shaded}

\hypertarget{descriptive-analysis}{%
\subsection{Descriptive Analysis}\label{descriptive-analysis}}

1. Conduct some descriptive analyses of these data. You can examine this
using scatter plots and correlation coefficients. Do you need to
transform any of the variables

\begin{Shaded}
\begin{Highlighting}[]
\FunctionTok{library}\NormalTok{(ggplot2)}


\FunctionTok{ggplot}\NormalTok{(main.data, }\FunctionTok{aes}\NormalTok{(}\AttributeTok{x =}\NormalTok{ mortality.rate)) }\SpecialCharTok{+}
  \FunctionTok{geom\_histogram}\NormalTok{(}\AttributeTok{bins =} \DecValTok{30}\NormalTok{, }\AttributeTok{fill =} \StringTok{"blue"}\NormalTok{, }\AttributeTok{color =} \StringTok{"black"}\NormalTok{) }\SpecialCharTok{+}
  \FunctionTok{ggtitle}\NormalTok{(}\StringTok{"Distribution of Mortality Rates"}\NormalTok{) }\SpecialCharTok{+}
  \FunctionTok{xlab}\NormalTok{(}\StringTok{"Mortality Rate"}\NormalTok{) }\SpecialCharTok{+}
  \FunctionTok{ylab}\NormalTok{(}\StringTok{"Frequency"}\NormalTok{)}
\end{Highlighting}
\end{Shaded}

\includegraphics{group-project-2_files/figure-latex/unnamed-chunk-5-1.pdf}

\begin{Shaded}
\begin{Highlighting}[]
\FunctionTok{ggplot}\NormalTok{(legalplus.main.data, }\FunctionTok{aes}\NormalTok{(}\AttributeTok{x =}\NormalTok{ Legality\_Status, }\AttributeTok{y =}\NormalTok{ mortality.rate, }\AttributeTok{fill =}\NormalTok{ Legality\_Status)) }\SpecialCharTok{+}
  \FunctionTok{geom\_boxplot}\NormalTok{() }\SpecialCharTok{+}
  \FunctionTok{ggtitle}\NormalTok{(}\StringTok{"Mortality Rate by Cannabis Legality Status"}\NormalTok{) }\SpecialCharTok{+}
  \FunctionTok{xlab}\NormalTok{(}\StringTok{"Cannabis Legality Status"}\NormalTok{) }\SpecialCharTok{+}
  \FunctionTok{ylab}\NormalTok{(}\StringTok{"Mortality Rate"}\NormalTok{)}
\end{Highlighting}
\end{Shaded}

\includegraphics{group-project-2_files/figure-latex/unnamed-chunk-5-2.pdf}

\begin{Shaded}
\begin{Highlighting}[]
\FunctionTok{ggplot}\NormalTok{(main.data, }\FunctionTok{aes}\NormalTok{(}\AttributeTok{x =}\NormalTok{ z.poverty, }\AttributeTok{y =}\NormalTok{ mortality.rate)) }\SpecialCharTok{+}
  \FunctionTok{geom\_point}\NormalTok{(}\AttributeTok{alpha =} \FloatTok{0.5}\NormalTok{) }\SpecialCharTok{+}
  \FunctionTok{geom\_smooth}\NormalTok{(}\AttributeTok{method =} \StringTok{"lm"}\NormalTok{, }\AttributeTok{se =} \ConstantTok{FALSE}\NormalTok{, }\AttributeTok{color =} \StringTok{"red"}\NormalTok{) }\SpecialCharTok{+}
  \FunctionTok{ggtitle}\NormalTok{(}\StringTok{"Mortality Rate vs. Poverty Rate"}\NormalTok{) }\SpecialCharTok{+}
  \FunctionTok{xlab}\NormalTok{(}\StringTok{"Below Poverty Rate"}\NormalTok{) }\SpecialCharTok{+}
  \FunctionTok{ylab}\NormalTok{(}\StringTok{"Mortality Rate"}\NormalTok{)}
\end{Highlighting}
\end{Shaded}

\begin{verbatim}
## `geom_smooth()` using formula = 'y ~ x'
\end{verbatim}

\includegraphics{group-project-2_files/figure-latex/unnamed-chunk-5-3.pdf}

\begin{Shaded}
\begin{Highlighting}[]
\FunctionTok{ggplot}\NormalTok{(main.data, }\FunctionTok{aes}\NormalTok{(}\AttributeTok{x =}\NormalTok{ z.highschool, }\AttributeTok{y =}\NormalTok{ mortality.rate)) }\SpecialCharTok{+}
  \FunctionTok{geom\_point}\NormalTok{(}\AttributeTok{alpha =} \FloatTok{0.5}\NormalTok{) }\SpecialCharTok{+}
  \FunctionTok{geom\_smooth}\NormalTok{(}\AttributeTok{method =} \StringTok{"lm"}\NormalTok{, }\AttributeTok{se =} \ConstantTok{FALSE}\NormalTok{, }\AttributeTok{color =} \StringTok{"blue"}\NormalTok{) }\SpecialCharTok{+}
  \FunctionTok{ggtitle}\NormalTok{(}\StringTok{"High School Rates vs. Mortality Rate"}\NormalTok{) }\SpecialCharTok{+}
  \FunctionTok{xlab}\NormalTok{(}\StringTok{"High School Rates"}\NormalTok{) }\SpecialCharTok{+}
  \FunctionTok{ylab}\NormalTok{(}\StringTok{"Mortality Rate"}\NormalTok{)}
\end{Highlighting}
\end{Shaded}

\begin{verbatim}
## `geom_smooth()` using formula = 'y ~ x'
\end{verbatim}

\includegraphics{group-project-2_files/figure-latex/unnamed-chunk-5-4.pdf}

\begin{Shaded}
\begin{Highlighting}[]
\FunctionTok{ggplot}\NormalTok{(main.data, }\FunctionTok{aes}\NormalTok{(}\AttributeTok{x =}\NormalTok{ z.opiates, }\AttributeTok{y =}\NormalTok{ mortality.rate)) }\SpecialCharTok{+}
  \FunctionTok{geom\_point}\NormalTok{(}\AttributeTok{alpha =} \FloatTok{0.5}\NormalTok{) }\SpecialCharTok{+}
  \FunctionTok{geom\_smooth}\NormalTok{(}\AttributeTok{method =} \StringTok{"lm"}\NormalTok{, }\AttributeTok{se =} \ConstantTok{FALSE}\NormalTok{, }\AttributeTok{color =} \StringTok{"green"}\NormalTok{) }\SpecialCharTok{+}
  \FunctionTok{ggtitle}\NormalTok{(}\StringTok{"Mortality Rate vs. Opiate use"}\NormalTok{) }\SpecialCharTok{+}
  \FunctionTok{xlab}\NormalTok{(}\StringTok{"Opiate use"}\NormalTok{) }\SpecialCharTok{+}
  \FunctionTok{ylab}\NormalTok{(}\StringTok{"Mortality Rate"}\NormalTok{)}
\end{Highlighting}
\end{Shaded}

\begin{verbatim}
## `geom_smooth()` using formula = 'y ~ x'
\end{verbatim}

\includegraphics{group-project-2_files/figure-latex/unnamed-chunk-5-5.pdf}

\begin{Shaded}
\begin{Highlighting}[]
\FunctionTok{ggplot}\NormalTok{(legalplus.main.data, }\FunctionTok{aes}\NormalTok{(}\AttributeTok{x =}\NormalTok{ Legality\_Status, }\AttributeTok{fill =}\NormalTok{ Legality\_Status)) }\SpecialCharTok{+}
  \FunctionTok{geom\_bar}\NormalTok{() }\SpecialCharTok{+}
  \FunctionTok{ggtitle}\NormalTok{(}\StringTok{"Number of Counties by Cannabis Legality Status"}\NormalTok{) }\SpecialCharTok{+}
  \FunctionTok{xlab}\NormalTok{(}\StringTok{"Cannabis Legality Status"}\NormalTok{) }\SpecialCharTok{+}
  \FunctionTok{ylab}\NormalTok{(}\StringTok{"Count"}\NormalTok{)}
\end{Highlighting}
\end{Shaded}

\includegraphics{group-project-2_files/figure-latex/unnamed-chunk-5-6.pdf}

\begin{Shaded}
\begin{Highlighting}[]
\FunctionTok{ggplot}\NormalTok{(legalplus.main.data, }\FunctionTok{aes}\NormalTok{(}\AttributeTok{x =}\NormalTok{ X2016.Prescribing.Rate, }\AttributeTok{y =}\NormalTok{ mortality.rate)) }\SpecialCharTok{+}
  \FunctionTok{geom\_point}\NormalTok{(}\FunctionTok{aes}\NormalTok{(}\AttributeTok{color =}\NormalTok{ Legality\_Status), }\AttributeTok{alpha =} \FloatTok{0.6}\NormalTok{) }\SpecialCharTok{+}
  \FunctionTok{geom\_smooth}\NormalTok{(}\AttributeTok{method =} \StringTok{"lm"}\NormalTok{, }\AttributeTok{se =} \ConstantTok{FALSE}\NormalTok{) }\SpecialCharTok{+}
  \FunctionTok{facet\_wrap}\NormalTok{(}\SpecialCharTok{\textasciitilde{}}\NormalTok{Legality\_Status) }\SpecialCharTok{+}
  \FunctionTok{ggtitle}\NormalTok{(}\StringTok{"Perscription Rates vs. Mortality Rate by Cannabis Legality Status"}\NormalTok{) }\SpecialCharTok{+}
  \FunctionTok{xlab}\NormalTok{(}\StringTok{"Perscription Rate"}\NormalTok{) }\SpecialCharTok{+}
  \FunctionTok{ylab}\NormalTok{(}\StringTok{"Mortality Rate"}\NormalTok{)}
\end{Highlighting}
\end{Shaded}

\begin{verbatim}
## `geom_smooth()` using formula = 'y ~ x'
\end{verbatim}

\includegraphics{group-project-2_files/figure-latex/unnamed-chunk-5-7.pdf}

\hypertarget{correlation-heatmap}{%
\paragraph{Correlation heatmap}\label{correlation-heatmap}}

\begin{Shaded}
\begin{Highlighting}[]
\NormalTok{numeric.data }\OtherTok{\textless{}{-}} \FunctionTok{data.frame}\NormalTok{(main.data}\SpecialCharTok{$}\NormalTok{z.poverty, main.data}\SpecialCharTok{$}\NormalTok{z.opiates, main.data}\SpecialCharTok{$}\NormalTok{mortality.rate, main.data}\SpecialCharTok{$}\NormalTok{z.highschool)}
\end{Highlighting}
\end{Shaded}

\begin{Shaded}
\begin{Highlighting}[]
\NormalTok{cormat }\OtherTok{\textless{}{-}} \FunctionTok{round}\NormalTok{(}\FunctionTok{cor}\NormalTok{(numeric.data),}\DecValTok{2}\NormalTok{)}
\FunctionTok{head}\NormalTok{(cormat)}
\end{Highlighting}
\end{Shaded}

\begin{verbatim}
##                          main.data.z.poverty main.data.z.opiates
## main.data.z.poverty                     1.00                0.32
## main.data.z.opiates                     0.32                1.00
## main.data.mortality.rate                0.22                0.24
## main.data.z.highschool                  0.72                0.20
##                          main.data.mortality.rate main.data.z.highschool
## main.data.z.poverty                          0.22                   0.72
## main.data.z.opiates                          0.24                   0.20
## main.data.mortality.rate                     1.00                   0.09
## main.data.z.highschool                       0.09                   1.00
\end{verbatim}

\begin{Shaded}
\begin{Highlighting}[]
\FunctionTok{library}\NormalTok{(reshape2)}
\end{Highlighting}
\end{Shaded}

\begin{verbatim}
## Warning: package 'reshape2' was built under R version 4.3.3
\end{verbatim}

\begin{verbatim}
## 
## Attaching package: 'reshape2'
\end{verbatim}

\begin{verbatim}
## The following object is masked from 'package:tidyr':
## 
##     smiths
\end{verbatim}

\begin{Shaded}
\begin{Highlighting}[]
\NormalTok{melted\_cormat }\OtherTok{\textless{}{-}} \FunctionTok{melt}\NormalTok{(cormat)}

\FunctionTok{ggplot}\NormalTok{(}\AttributeTok{data =}\NormalTok{ melted\_cormat, }\FunctionTok{aes}\NormalTok{(Var2, Var1, }\AttributeTok{fill =}\NormalTok{ value))}\SpecialCharTok{+}
 \FunctionTok{geom\_tile}\NormalTok{(}\AttributeTok{color =} \StringTok{"white"}\NormalTok{)}\SpecialCharTok{+}
 \FunctionTok{scale\_fill\_gradient2}\NormalTok{(}\AttributeTok{low =} \StringTok{"blue"}\NormalTok{, }\AttributeTok{high =} \StringTok{"red"}\NormalTok{, }\AttributeTok{mid =} \StringTok{"white"}\NormalTok{, }
   \AttributeTok{midpoint =} \DecValTok{0}\NormalTok{, }\AttributeTok{limit =} \FunctionTok{c}\NormalTok{(}\SpecialCharTok{{-}}\DecValTok{1}\NormalTok{,}\DecValTok{1}\NormalTok{), }\AttributeTok{space =} \StringTok{"Lab"}\NormalTok{, }
   \AttributeTok{name=}\StringTok{"Pearson}\SpecialCharTok{\textbackslash{}n}\StringTok{Correlation"}\NormalTok{) }\SpecialCharTok{+}
  \FunctionTok{theme\_minimal}\NormalTok{()}\SpecialCharTok{+} 
 \FunctionTok{theme}\NormalTok{(}\AttributeTok{axis.text.x =} \FunctionTok{element\_text}\NormalTok{(}\AttributeTok{angle =} \DecValTok{45}\NormalTok{, }\AttributeTok{vjust =} \DecValTok{1}\NormalTok{, }
    \AttributeTok{size =} \DecValTok{12}\NormalTok{, }\AttributeTok{hjust =} \DecValTok{1}\NormalTok{))}\SpecialCharTok{+}
 \FunctionTok{coord\_fixed}\NormalTok{()}
\end{Highlighting}
\end{Shaded}

\includegraphics{group-project-2_files/figure-latex/unnamed-chunk-8-1.pdf}

\hypertarget{relationship-of-dependent-and-independent-variables}{%
\subsection{Relationship of dependent and independent
variables}\label{relationship-of-dependent-and-independent-variables}}

To explore the question `How does poverty and opiate prescriptions
predict mortality rates associated with drugs and alcohol?' we have
explored potential variables including poverty rates, opiate
prescription rates, education levels and state laws for drug use. With a
numerical dependent variable, mortality rate, a linear regression type
may be suitable to model the relationship.

A few assumptions need to be made in order to use this model:

Linearity: The relationship between the predictor variables (poverty,
opioid prescriptions) and the response variable (mortality rates
associated with drugs and alcohol) should be approximately linear. This
assumption is assessed through our descriptive analysis scatter plots.

Independence: The observations should be independent of each other. In
other words, the mortality rates for one county should not influence the
mortality rates for another county.

Homoscedasticity: The variance of the residuals should be constant
across all levels of the predictors. This can be checked by examining a
plot of residuals versus fitted values.

Normality of Residuals: The residuals should follow a normal
distribution. This assumption can be assessed by examining a histogram
or a Q-Q plot of the residuals.

No Multicollinearity: The predictor variables should not be highly
correlated with each other. Multicollinearity can inflate standard
errors and make interpretation of coefficients difficult. This is seen
in the correlation matrix and it can be assumed this assumption is met.

Make sure to drop the NA values for all the other columns:

\begin{Shaded}
\begin{Highlighting}[]
\NormalTok{legalplus.main.data }\OtherTok{=} \FunctionTok{drop\_na}\NormalTok{(legalplus.main.data, mortality.rate, z.poverty, z.opiates, z.highschool, Legality\_Status)}
\end{Highlighting}
\end{Shaded}

Model including just the standardised poverty and opiate variables.

\begin{Shaded}
\begin{Highlighting}[]
\NormalTok{model1 }\OtherTok{\textless{}{-}} \FunctionTok{lm}\NormalTok{(mortality.rate }\SpecialCharTok{\textasciitilde{}}\NormalTok{ z.poverty }\SpecialCharTok{+}\NormalTok{ z.opiates, }\AttributeTok{data =}\NormalTok{ main.data)}
\FunctionTok{summary}\NormalTok{(model1)}
\end{Highlighting}
\end{Shaded}

\begin{verbatim}
## 
## Call:
## lm(formula = mortality.rate ~ z.poverty + z.opiates, data = main.data)
## 
## Residuals:
##     Min      1Q  Median      3Q     Max 
## -20.983  -5.824  -1.562   3.963  91.399 
## 
## Coefficients:
##             Estimate Std. Error t value Pr(>|t|)    
## (Intercept)  20.4864     0.1757 116.591   <2e-16 ***
## z.poverty     1.6040     0.1941   8.265   <2e-16 ***
## z.opiates     1.8630     0.1916   9.724   <2e-16 ***
## ---
## Signif. codes:  0 '***' 0.001 '**' 0.01 '*' 0.05 '.' 0.1 ' ' 1
## 
## Residual standard error: 8.934 on 2609 degrees of freedom
## Multiple R-squared:  0.08339,    Adjusted R-squared:  0.08269 
## F-statistic: 118.7 on 2 and 2609 DF,  p-value: < 2.2e-16
\end{verbatim}

Model includes the standardised numerical variables and state law types.

\begin{Shaded}
\begin{Highlighting}[]
\NormalTok{regmodel }\OtherTok{\textless{}{-}} \FunctionTok{lm}\NormalTok{(mortality.rate }\SpecialCharTok{\textasciitilde{}}\NormalTok{ z.poverty }\SpecialCharTok{+}\NormalTok{ z.opiates }\SpecialCharTok{+}\NormalTok{ z.highschool  }\SpecialCharTok{+}\NormalTok{ Legality\_Status, }\AttributeTok{data =}\NormalTok{ legalplus.main.data)}

\FunctionTok{summary}\NormalTok{(regmodel)}
\end{Highlighting}
\end{Shaded}

\begin{verbatim}
## 
## Call:
## lm(formula = mortality.rate ~ z.poverty + z.opiates + z.highschool + 
##     Legality_Status, data = legalplus.main.data)
## 
## Residuals:
##     Min      1Q  Median      3Q     Max 
## -27.215  -5.681  -1.396   3.782  88.084 
## 
## Coefficients:
##                                                   Estimate Std. Error t value
## (Intercept)                                        17.0504     1.4964  11.394
## z.poverty                                           2.6969     0.2679  10.065
## z.opiates                                           1.9514     0.1871  10.428
## z.highschool                                       -1.1065     0.2757  -4.014
## Legality_StatusIllegal                              7.3634     1.9039   3.868
## Legality_StatusLegal for Medical and Recreational   5.9229     1.5222   3.891
## Legality_StatusLegal for Medical Only               2.0645     1.5142   1.363
##                                                   Pr(>|t|)    
## (Intercept)                                        < 2e-16 ***
## z.poverty                                          < 2e-16 ***
## z.opiates                                          < 2e-16 ***
## z.highschool                                      6.14e-05 ***
## Legality_StatusIllegal                            0.000113 ***
## Legality_StatusLegal for Medical and Recreational 0.000102 ***
## Legality_StatusLegal for Medical Only             0.172873    
## ---
## Signif. codes:  0 '***' 0.001 '**' 0.01 '*' 0.05 '.' 0.1 ' ' 1
## 
## Residual standard error: 8.692 on 2605 degrees of freedom
## Multiple R-squared:  0.1337, Adjusted R-squared:  0.1317 
## F-statistic:    67 on 6 and 2605 DF,  p-value: < 2.2e-16
\end{verbatim}

Testing Homoscedasticity: There is no real pattern in the plot which is
good for homoscedasticity. The spread is quite random.

\begin{Shaded}
\begin{Highlighting}[]
\NormalTok{residuals }\OtherTok{\textless{}{-}} \FunctionTok{resid}\NormalTok{(model1)}
\NormalTok{fitted\_values }\OtherTok{\textless{}{-}} \FunctionTok{fitted}\NormalTok{(model1)}
\FunctionTok{plot}\NormalTok{(fitted\_values, residuals, }\AttributeTok{main =} \StringTok{"Residuals vs Fitted Values"}\NormalTok{, }
     \AttributeTok{xlab =} \StringTok{"Fitted Values"}\NormalTok{, }\AttributeTok{ylab =} \StringTok{"Residuals"}\NormalTok{)}
\FunctionTok{abline}\NormalTok{(}\AttributeTok{h =} \DecValTok{0}\NormalTok{, }\AttributeTok{col =} \StringTok{"red"}\NormalTok{)  }
\end{Highlighting}
\end{Shaded}

\includegraphics{group-project-2_files/figure-latex/unnamed-chunk-12-1.pdf}
Testing for Normality of Residuals: the plotted points don't form a
straight line and indicate that the residuals deviate slightly and the
tail ends suggest a skewness.

\begin{Shaded}
\begin{Highlighting}[]
\FunctionTok{qqnorm}\NormalTok{(residuals)}
\FunctionTok{qqline}\NormalTok{(residuals)}
\end{Highlighting}
\end{Shaded}

\includegraphics{group-project-2_files/figure-latex/unnamed-chunk-13-1.pdf}

\hypertarget{interpretations}{%
\subsubsection{Interpretations}\label{interpretations}}

We have made two models to identify if the inclusion of confounding
factors better explains the mortality rate variable.

Model 1 identifies that the linear regression model can be suitable for
predicting mortality rates with the assumptions holding. The
coefficients in this model are all statistically significant.

\hypertarget{model-fit}{%
\subsubsection{Model Fit}\label{model-fit}}

\hypertarget{map}{%
\subsection{Map}\label{map}}

3. Create at least one map that helps explain your findings (we will
show you how to do this in the next session).

\begin{Shaded}
\begin{Highlighting}[]
\NormalTok{us.shape}\FloatTok{.2} \OtherTok{\textless{}{-}}\NormalTok{ us.shape}\FloatTok{.2} \SpecialCharTok{\%\textgreater{}\%}
  \FunctionTok{shift\_geometry}\NormalTok{()}
\end{Highlighting}
\end{Shaded}

\begin{Shaded}
\begin{Highlighting}[]
\CommentTok{\# Create the ggplot object}
\NormalTok{p1 }\OtherTok{\textless{}{-}} \FunctionTok{ggplot}\NormalTok{(us.shape}\FloatTok{.2}\NormalTok{) }\SpecialCharTok{+}
  \FunctionTok{geom\_sf}\NormalTok{(}\AttributeTok{mapping =} \FunctionTok{aes}\NormalTok{(}\AttributeTok{fill =}\NormalTok{ mortality.rate), }\AttributeTok{colour =} \ConstantTok{NA}\NormalTok{, }\AttributeTok{inherit.aes =} \ConstantTok{FALSE}\NormalTok{) }\SpecialCharTok{+}
  \FunctionTok{scale\_fill\_gradient2}\NormalTok{(}\AttributeTok{low =} \StringTok{"dark blue"}\NormalTok{,}
                       \AttributeTok{high =} \StringTok{"dark green"}\NormalTok{,}
                       \AttributeTok{midpoint =} \FunctionTok{mean}\NormalTok{(us.shape}\FloatTok{.2}\SpecialCharTok{$}\NormalTok{mortality.rate, }\AttributeTok{na.rm =} \ConstantTok{TRUE}\NormalTok{),}
                       \AttributeTok{limits =} \FunctionTok{c}\NormalTok{(}\FunctionTok{min}\NormalTok{(us.shape}\FloatTok{.2}\SpecialCharTok{$}\NormalTok{mortality.rate, }\AttributeTok{na.rm =} \ConstantTok{TRUE}\NormalTok{),}
                                  \FunctionTok{max}\NormalTok{(us.shape}\FloatTok{.2}\SpecialCharTok{$}\NormalTok{mortality.rate, }\AttributeTok{na.rm =} \ConstantTok{TRUE}\NormalTok{)),}
                       \AttributeTok{label =}\NormalTok{ scales}\SpecialCharTok{::}\NormalTok{comma,}
                       \AttributeTok{na.value =} \StringTok{"grey"}\NormalTok{) }\SpecialCharTok{+}
  \FunctionTok{labs}\NormalTok{(}\AttributeTok{title =} \StringTok{\textquotesingle{}Mortality Rate distribution\textquotesingle{}}\NormalTok{) }\SpecialCharTok{+}
  \FunctionTok{guides}\NormalTok{(}\AttributeTok{fill =} \FunctionTok{guide\_legend}\NormalTok{(}\AttributeTok{keyheight =} \FloatTok{0.3}\NormalTok{)) }\SpecialCharTok{+}
  \FunctionTok{theme\_void}\NormalTok{() }\SpecialCharTok{+}
  \FunctionTok{theme}\NormalTok{(}\AttributeTok{plot.title =} \FunctionTok{element\_text}\NormalTok{(}\AttributeTok{size =} \DecValTok{12}\NormalTok{),}
        \AttributeTok{legend.position =} \StringTok{\textquotesingle{}bottom\textquotesingle{}}\NormalTok{,}
        \AttributeTok{legend.title =} \FunctionTok{element\_blank}\NormalTok{())}

\CommentTok{\# options(repr.plot.width = 12, repr.plot.height = 10)  \# Adjust width and height as needed}
\CommentTok{\# p1  \# Display the plot}
\end{Highlighting}
\end{Shaded}

\begin{Shaded}
\begin{Highlighting}[]
\CommentTok{\# Create the ggplot object}
\NormalTok{p2 }\OtherTok{\textless{}{-}} \FunctionTok{ggplot}\NormalTok{(us.shape}\FloatTok{.2}\NormalTok{) }\SpecialCharTok{+}
  \FunctionTok{geom\_sf}\NormalTok{(}\AttributeTok{mapping =} \FunctionTok{aes}\NormalTok{(}\AttributeTok{fill =}\NormalTok{ below.poverty.rate), }\AttributeTok{colour =} \ConstantTok{NA}\NormalTok{, }\AttributeTok{inherit.aes =} \ConstantTok{FALSE}\NormalTok{) }\SpecialCharTok{+}
  \FunctionTok{scale\_fill\_gradient2}\NormalTok{(}\AttributeTok{low =} \StringTok{"dark blue"}\NormalTok{,}
                       \AttributeTok{high =} \StringTok{"dark green"}\NormalTok{,}
                       \AttributeTok{midpoint =} \FunctionTok{mean}\NormalTok{(us.shape}\FloatTok{.2}\SpecialCharTok{$}\NormalTok{below.poverty.rate, }\AttributeTok{na.rm =} \ConstantTok{TRUE}\NormalTok{),}
                       \AttributeTok{limits =} \FunctionTok{c}\NormalTok{(}\FunctionTok{min}\NormalTok{(us.shape}\FloatTok{.2}\SpecialCharTok{$}\NormalTok{below.poverty.rate, }\AttributeTok{na.rm =} \ConstantTok{TRUE}\NormalTok{),}
                                  \FunctionTok{max}\NormalTok{(us.shape}\FloatTok{.2}\SpecialCharTok{$}\NormalTok{below.poverty.rate, }\AttributeTok{na.rm =} \ConstantTok{TRUE}\NormalTok{)),}
                       \AttributeTok{label =}\NormalTok{ scales}\SpecialCharTok{::}\NormalTok{comma,}
                       \AttributeTok{na.value =} \StringTok{"grey"}\NormalTok{) }\SpecialCharTok{+}
  \FunctionTok{labs}\NormalTok{(}\AttributeTok{title =} \StringTok{\textquotesingle{}Mortality Rate distribution\textquotesingle{}}\NormalTok{) }\SpecialCharTok{+}
  \FunctionTok{guides}\NormalTok{(}\AttributeTok{fill =} \FunctionTok{guide\_legend}\NormalTok{(}\AttributeTok{keyheight =} \FloatTok{0.3}\NormalTok{)) }\SpecialCharTok{+}
  \FunctionTok{theme\_void}\NormalTok{() }\SpecialCharTok{+}
  \FunctionTok{theme}\NormalTok{(}\AttributeTok{plot.title =} \FunctionTok{element\_text}\NormalTok{(}\AttributeTok{size =} \DecValTok{12}\NormalTok{),}
        \AttributeTok{legend.position =} \StringTok{\textquotesingle{}bottom\textquotesingle{}}\NormalTok{,}
        \AttributeTok{legend.title =} \FunctionTok{element\_blank}\NormalTok{())}

\CommentTok{\# options(repr.plot.width = 12, repr.plot.height = 10)  \# Adjust width and height as needed}
\CommentTok{\# p2  \# Display the plot}
\end{Highlighting}
\end{Shaded}

\begin{Shaded}
\begin{Highlighting}[]
\CommentTok{\# Create the ggplot object}
\NormalTok{p3 }\OtherTok{\textless{}{-}} \FunctionTok{ggplot}\NormalTok{(us.shape}\FloatTok{.2}\NormalTok{) }\SpecialCharTok{+}
  \FunctionTok{geom\_sf}\NormalTok{(}\AttributeTok{mapping =} \FunctionTok{aes}\NormalTok{(}\AttributeTok{fill =}\NormalTok{ X2016.Prescribing.Rate), }\AttributeTok{colour =} \ConstantTok{NA}\NormalTok{, }\AttributeTok{inherit.aes =} \ConstantTok{FALSE}\NormalTok{) }\SpecialCharTok{+}
  \CommentTok{\# coord\_sf(xlim = c({-}180, {-}65), ylim = c(24, 72), expand = FALSE) +}
  \FunctionTok{scale\_fill\_gradient2}\NormalTok{(}\AttributeTok{low =} \StringTok{"dark blue"}\NormalTok{,}
                       \AttributeTok{high =} \StringTok{"dark green"}\NormalTok{,}
                       \AttributeTok{midpoint =} \FunctionTok{mean}\NormalTok{(us.shape}\FloatTok{.2}\SpecialCharTok{$}\NormalTok{X2016.Prescribing.Rate, }\AttributeTok{na.rm =} \ConstantTok{TRUE}\NormalTok{),}
                       \AttributeTok{limits =} \FunctionTok{c}\NormalTok{(}\FunctionTok{min}\NormalTok{(us.shape}\FloatTok{.2}\SpecialCharTok{$}\NormalTok{X2016.Prescribing.Rate, }\AttributeTok{na.rm =} \ConstantTok{TRUE}\NormalTok{),}
                                  \FunctionTok{max}\NormalTok{(us.shape}\FloatTok{.2}\SpecialCharTok{$}\NormalTok{X2016.Prescribing.Rate, }\AttributeTok{na.rm =} \ConstantTok{TRUE}\NormalTok{)),}
                       \AttributeTok{label =}\NormalTok{ scales}\SpecialCharTok{::}\NormalTok{comma,}
                       \AttributeTok{na.value =} \StringTok{"grey"}\NormalTok{) }\SpecialCharTok{+}
  \FunctionTok{labs}\NormalTok{(}\AttributeTok{title =} \StringTok{\textquotesingle{}Mortality Rate distribution\textquotesingle{}}\NormalTok{) }\SpecialCharTok{+}
  \FunctionTok{guides}\NormalTok{(}\AttributeTok{fill =} \FunctionTok{guide\_legend}\NormalTok{(}\AttributeTok{keyheight =} \FloatTok{0.3}\NormalTok{)) }\SpecialCharTok{+}
  \FunctionTok{theme\_void}\NormalTok{() }\SpecialCharTok{+}
  \FunctionTok{theme}\NormalTok{(}\AttributeTok{plot.title =} \FunctionTok{element\_text}\NormalTok{(}\AttributeTok{size =} \DecValTok{12}\NormalTok{),}
        \AttributeTok{legend.position =} \StringTok{\textquotesingle{}bottom\textquotesingle{}}\NormalTok{,}
        \AttributeTok{legend.title =} \FunctionTok{element\_blank}\NormalTok{())}

\CommentTok{\# options(repr.plot.width = 12, repr.plot.height = 10)  \# Adjust width and height as needed}
\CommentTok{\# p3  \# Display the plot}
\end{Highlighting}
\end{Shaded}

\begin{Shaded}
\begin{Highlighting}[]
\NormalTok{p1 }\SpecialCharTok{+}\NormalTok{p2 }\SpecialCharTok{+}\NormalTok{ p3 }\SpecialCharTok{+}
  \FunctionTok{plot\_annotation}\NormalTok{(}\AttributeTok{title =} \StringTok{\textquotesingle{}Demographic distribution in America\textquotesingle{}}\NormalTok{) }\SpecialCharTok{+}
  \FunctionTok{plot\_layout}\NormalTok{(}\AttributeTok{ncol =} \DecValTok{3}\NormalTok{)}
\end{Highlighting}
\end{Shaded}

\includegraphics{group-project-2_files/figure-latex/unnamed-chunk-18-1.pdf}

\hypertarget{discussion}{%
\subsection{Discussion}\label{discussion}}

4. Write up your key results.

You will be marked on the quality of your \$R\$ code (including whether
it runs for us without errors), how well you have justified your
variable selection, the proper use of appropriate methods.

\end{document}
